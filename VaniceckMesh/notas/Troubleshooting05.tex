\documentclass[a4paper,12pt]{article}

\usepackage[utf8]{inputenc}
%\usepackage[spanish]{babel}
\usepackage{amsmath,amssymb}
\usepackage{graphicx}
%\usepackage{subfigure}
\usepackage{bm}
\usepackage{caption}
\usepackage{subcaption}

\usepackage[cm]{fullpage}
\usepackage[light]{antpolt}
\usepackage[T1]{fontenc}

\usepackage{float}

%Este paquete le pone una barrera a los floats
% al final de cada seccion
\usepackage[section]{placeins}


\bibliographystyle{alpha}

\newcommand{\dt}{\Delta\tau}
\newcommand{\ihb}{\frac{i}{\hbar}}
\newcommand{\xfase}{\mathbf{x}}
\newcommand{\yfase}{\mathbf{y}}
\newcommand{\qfase}{\mathbf{q}}
\newcommand{\pfase}{\mathbf{p}}
\newcommand{\xifase}{ {\boldsymbol{\xi}} }
\newcommand{\mufase}{ {\boldsymbol{\mu}} }
\newcommand{\Ifase}{\mathbf{I}}
\newcommand{\Pfase}{\mathbf{P}}
\newcommand{\Scat}{\mathbf{S}}
\newcommand{\Jsimp}{\mathbf{J}}
\newcommand{\Dom}{\mathbb{D}}
\newcommand{\Var}{\mathbb{M}}
\newcommand{\bra}[1]{\langle #1|}
\newcommand{\ket}[1]{|#1\rangle}
\newcommand{\braket}[2]{\langle #1|#2\rangle}
\newcommand{\Prom}[2]{\langle #1\rangle_{#2}}
\newcommand{\rd}{\!\mathrm{d}}

\DeclareMathOperator*{\cod}{cod}
\DeclareMathOperator*{\traza}{traza}

\begin{document}

\title{Troubleshooting 05: Producing hundreds of Eigenstates. }
\author{Zambrano et Zapfe}


\maketitle

\section{Hernando and Zambrano's Method}

Recently, Hernando and Vaniceck published a method for obtaining
very good numerical representation of Eigenfunctions for bounded
quantum states \cite{Hernando}. The method is so simple
that it seems a rip-off, but it is not. 
Before tackling the problem with the Nelson Potential we (Zambrano and Karel)
decided to test it first. When we got back Hernando and Vaniceck 
results, we decided to use it for the Nelson Potential.
Here is a brief explanation of the method and the
modification proposed by Eduardo.

Let us set things straight and formal.

The formal solution for \emph{any} state represented by a state
function $\Psi(q,t)$ under temporal evolution is given by:
\begin{equation}
\Psi(q,t)=\int dq' K(q,q', t-t') \Psi(q',t'),
\end{equation}
We set  $t_0=0$.

Last equation is just syntactic sugar for the more formal expression below:
\begin{equation}\label{evolbien}
\bra{q}\ket{\Psi_t}=\int dq' \bra{q}U^t\ket{q'}\bra{q'}\ket{\Psi_0},
\end{equation}
where we have expanded the identity in the  $q'$: 
\begin{equation}
\Ifase=\int dq' \ket{q'}\bra{q'}.
\end{equation}
The vectorial nature of this formalism is more evident
if we use abstract vectors instead of position space functions:
\begin{equation}
\ket{\Psi_t}=\int dq' U^t\ket{q'}\bra{q'}\ket{\Psi_0},
\end{equation}

Formally the evolution operator is:
\begin{equation}
U^t=\exp{(-\ihb \hat{H} t)} 
\end{equation}
If the Hamiltonian has the 
usual form  $\hat{H}=\hat{T}+\hat{V}$, we could approximate the
evolution operator by: 
\begin{equation}\label{aproxHernando}
\exp(-\ihb \hat{H} t)\approx 
 \exp(-\ihb \hat{T} t)\exp(-\ihb \hat{V}t)+O([T,V]^2 t^2).
\end{equation}
Eduardo's modification consisted in using the following
expression, which is numerically as simple as the
former, but a bit better on the convergence for small $t$:
\begin{equation}\label{aproxEduardo}
\exp(-\ihb \hat{H} t)\approx
 \exp(-\ihb \hat{V}t/2)\exp(-\ihb \hat{T} t)\exp(-\ihb \hat{V}t/2)+O([T,V]^4 t^4).
\end{equation}
We take this last expression and use it on 
\ref{evolbien}:
\begin{equation}\label{ketaprox1}
\bra{q}\ket{\Psi_t}  = \int dq' \bra{q} 
\exp(-\ihb \hat{V}t/2)\exp(-\ihb \hat{T} t)\exp(-\ihb \hat{V}t/) 
\ket{q'}\bra{q'}\ket{\Psi_0},
\end{equation}
If $\exp(-\ihb \hat{V}t/2)$ is a well behaved function of  $\hat{q}$,
then it produces the usual result over the  $q$ Eigenstates, that
is, it produces a function of the eigenvalues multiplying the
eigenstate:
\begin{equation}\label{ketaprox2}
  \bra{q}\ket{\Psi_t}  = \int dq' \exp(-\ihb t/2 (V(q)+V(q'))) 
  \bra{q} \exp{-\ihb \hat{T} t}\ket{q'}\bra{q'}\ket{\Psi_0}.
\end{equation}
What stays inside the brackets is the propagator for a free
particle, which gives:
\begin{equation}\label{ketaprox2}
  \bra{q}\ket{\Psi_t}  = \int dq' \exp(-\ihb t/2 (V(q)+V(q'))) 
  (\frac{ i }{2\pi\hbar t})^{d/2} \exp(\ihb 1/(2t) (q-q')^2)
  \bra{q'}\ket{\Psi_0}.
\end{equation}

Until here, nothing is new. Now we follow Hernando and use an imaginary ``time''
of small magnitude, $t=-i\dt$, and
$\Psi(q,t)$ representing a stationary state, i.e. 
$\Psi(q,t)=\exp(-\ihb E_n t)\psi_n(q)$.
\begin{equation}\label{ketaprox2}
  \exp\big(-\frac{E_n \dt}{\hbar}\big) \psi_n(q_k)     
= \int dq' \exp[-\hbar \dt/2 (V(q)+V(q'))] 
  (\frac{1}{2\pi\hbar \dt})^{d/2} \exp[- \hbar/(2t) (q-q')^2]
  \psi_n(q').
\end{equation}

In our scheme, we shall use an uniform grid of points in $q$ space. Then our
identity operator takes the approximate form:
\begin{equation}
\int dq' \ket{q'}\bra{q'}\approx \frac{A}{N}\sum_j \ket{q_j'}\bra{q_j'}.
\end{equation}
Where $N$ is the number of points in the grid
and $A$ the area covered by the sampling. 

This shall then produce the following approximate equation for
the Eigenstates:
\begin{equation}\label{ketaprox3}
 \exp\big[-\frac{E_n \dt}{\hbar}\big] = \frac{A}{N}\sum_j 
\exp[-\hbar \dt/2 (V(q_k)+V(q_j))] 
  (\frac{1}{2\pi\hbar \dt})^{d/2} \exp[- \hbar/(2t) (q_k-q_j)^2]
  \psi_n(q_j).
\end{equation}
which is clearly an eigenvalue equation for the operator:
\begin{equation}\label{opK}
\tilde{K}_{kj}=\frac{A}{N}
\exp(-\hbar \dt/2 (V(q_k)+V(q_j))) 
  (\frac{1}{2\pi\hbar \dt})^{d/2} \exp(- \hbar/(2\dt) (q_k-q_j)^2)
\end{equation}

The problem then is really giving the eigensystem solution
for the expression \ref{ketaprox2} viewed as:
\begin{equation}\label{eigenv}
  \exp(\frac{- E_n \dt}{\hbar}) \psi_n  = K \psi_n.
\end{equation}
The points in the grid act as indexes for the eigenvector 
components.


Could this be used for solving a numerical approximation for the
Eigenstates and Eigenvalues? According to \cite{Hernando} is
clear that it can be done. We could easily obtain
their results by translating their Mathematica code to our
own languages and we could also reproduce them after
Eduardo's modification of the idea without mayor errors,
so we are confident in the method. The same numerical caveats 
pointed in the papers occur, no matter if we follow the
method exactly or our own version. The method shows an
uncomfortable dependency on the size of $\dt$, and
only the lowest $1/5$th part of the Eigenvectors
thus obtained are trustworthy. 

It is my suggestion
to use only open-sourced resources that are well established on
the ``scientific numerical'' community, so I translated all of
this to a mixture of \verb!c++! and Python code. People who
developed the linear algebra routines that we use here
ask to be cited so that their effort is noticed, so 
(if this ever gets published) shall give adequate
acknowledgement of this.

\section{Technical details on the procedure}

The first caveat is the problem of finding an adequate grid.
The Nelson potential is a 2 d.o.f. problem, stated here as:
\begin{equation}
V(q_1, q_2)=q_1^2/20+(q_2-q_1^2/2)^2.
\end{equation}
Even for 
very low resolutions, the operator $K$ in the equation \ref{opK}
grows very fast, consuming a lot of memory space. Let us say that
we put $N_1$ points per side on the $q$-space, then the eigenvector
resolution (that is, the number of dimensions of the eigenvector),
is $N_1^2=N$ and the operator $K$ has $N_1^4$ entries.
A mesh of $N_1=100$ puts the limit of Python on my machine to test,
but it works for checking the general behaviour of the algorithm. 

According to the old work by Baranger et al. \cite{Bar87, Bar93},
the Nelson Potential shows already very chaotic behaviour for relative low
energies, of the order of $0.1$ in the units used. It must be said
that the parameters chosen there have remained untouched along 
all the work from there, to F. Toscano, and to mine. The 
frequencies parameters, energy ranges and value for $\hbar$ is the same for
everyone. For energy ranges around $0.8$, the classical trajectories
are bounded by a caustic that looks like a quarter moon, banana or boomerang
in the $q$-space which can be covered by a square with the corners given 
by the next set of parameters:

\begin{align}
q_{1,min} =- 5.01, &\;  q_{1,max} = 5.01 \\
q_{2,min} =- 1.01, &\;  q_{2,max} = 9.01.
\end{align}

Our grid is uniform, so we do not use the reduction proposed by Hernando.
In the first ``pythonic'' try, we command the computer to use a mesh of 
$N=80^2$ points, $80$ by side, and a $ \dt=0.1875$. In personal communication
with Vaniceck he suggested us to use $\dt \approx \Delta q ^2$, where
$\Delta q$ is the spacing in the grid. Hernando suggested Eduardo that
for our approximation to the propagator a good rule of thumb for 
the $\dt$ magnitude is to be around this expression:
\begin{equation}
\frac{m \Delta q^2}{\hbar \dt}\sim 1.
\end{equation}
The value that I encountered is a bit more than 
 half of the value
required for the last expression to be an equality.
The lower bounds on the paper are respected by this expression.
Also, it must be said that playing around with the values of
$\dt$, the shape of the Eigenstates remains robust, only a
slight change in the energies is observed. 
Our algorithms first produces the uniform grid of points in
the mentioned square. Then we calculate entry by entry of the
matrix of the propagator $K$ as given by the equation 
\ref{opk}. We use the numerical routines of SciPy \cite{scipy},
which has a powerful linear algebra routine for solving Eigenstates.
It must be said that we do not impose a limit on the precision, so
we use the full capacity of the machine for solving eigenvalues.

Using this method we have obtained around 350 thrust-worthy Eigenvectors, 
with a resolution of 80 by 80 points. which I show in the figure
\ref{muestrasprimerintento}.

\begin{figure}[h]
  \centering
  \begin{subfigure}[b]{0.48\textwidth}
    \includegraphics[width=\textwidth]{figuras/EstadoLowRes001.png}
    \caption{state 1}
    \label{fig:gull}
  \end{subfigure}%
  % add desired spacing between images, e. g. ~, \quad, \qquad, \hfill etc.
  % (or a blank line to force the subfigure onto a new line)
  \begin{subfigure}[b]{0.48\textwidth}
    \includegraphics[width=\textwidth]{figuras/EstadoLowRes073.png}
    \caption{state 73}
    \label{fig:tiger}
  \end{subfigure}\\
  % add desired spacing between images, e. g. ~, \quad, \qquad, \hfill etc.
  % (or a blank line to force the subfigure onto a new line)
  \begin{subfigure}[b]{0.48\textwidth}
    \includegraphics[width=\textwidth]{figuras/EstadoLowRes200.png}
    \caption{state 200}
    \label{fig:mouse}
  \end{subfigure}
  \begin{subfigure}[b]{0.48\textwidth}
    \includegraphics[width=\textwidth]{figuras/EstadoLowRes294.png}
    \caption{state 294}
    \label{fig:mouse}
  \end{subfigure}
  \caption{Some Eigenfunctions as obtained by the Hernando-Zambrano method.} 
  \label{muestrasprimerintento}
\end{figure}

The resolution of the figures is very crude, but it suffices to
see the growing of the horns of the bananoid, and the ``scarification'' process
as we go to higher energies. 

Still, we cannot use this data for obtaining the centre and chord
functions, precisely because our resolution is to low and at the
energies that we are interested ($E\sim 0.8$) the oscillations are already
very dense. 

For obtaining a slightly more accurate version of the Eigenfunctions,
I have translated the routine to a lower level language, namely,
\verb!c++!, and used the Armadillo Linear Algebra library, with bindings to
Lapack \cite{armadillo}. This has proven much faster and I could force the
machine for a denser grid (3 times more dense). Armadillo has a very
powerful routine which enables it to calculate only the lower or higher
eigenvalues of a sparse matrix. Our propagator $K$ is \emph{almost}
a sparse matrix, as the magnitude of the elements decreases as a gaussian
outside the diagonal. It turns out that we can set a very low numerical
tolerance and set the values of $K_{jk}=0$ if their absolute value is
below that tolerance. I have chosen the tolerance to be 
$\epsilon=10^{-5}$, and this gives consistent results with
the ScyPy routine but produces denser Eigenvectors, and in the
sense explained in \cite{Hernando}, also more confidence 
on the higher levels. The next figure, \ref{muestrasintentomejor},
shows the Eigenfunctions in $q$-space thus obtained. The number of points
by side is $256$, and I have settled for $\dt=0.03$. This puts the 
parameters in the right ballpark.


\begin{figure}[h]
  \centering
  \begin{subfigure}[b]{0.48\textwidth}
    \includegraphics[width=\textwidth]{figuras/EstadoSemiRes001.png}
    \caption{state 1}
    \label{fig:gull}
  \end{subfigure}%
  % add desired spacing between images, e. g. ~, \quad, \qquad, \hfill etc.
  % (or a blank line to force the subfigure onto a new line)
  \begin{subfigure}[b]{0.48\textwidth}
    \includegraphics[width=\textwidth]{figuras/EstadoSemiRes073.png}
    \caption{state 73}
    \label{fig:tiger}
  \end{subfigure}\\
  % add desired spacing between images, e. g. ~, \quad, \qquad, \hfill etc.
  % (or a blank line to force the subfigure onto a new line)
  \begin{subfigure}[b]{0.48\textwidth}
    \includegraphics[width=\textwidth]{figuras/EstadoSemiRes200.png}
    \caption{state 200}
    \label{fig:mouse}
  \end{subfigure}
  \begin{subfigure}[b]{0.48\textwidth}
    \includegraphics[width=\textwidth]{figuras/EstadoSemiRes294.png}
    \caption{state 294}
    \label{fig:mouse}
  \end{subfigure}
  \caption{Some Eigenfunctions as obtained by the Hernando-Zambrano method,
this time with Armadillo and Lapack on a  c++   
program. Much better resolution
and clearer definition of the scars and highs and lows is notorious, also, the
Eigenergies are closer as those reported by previous works. Structural differences 
are allready noticable for the higher state. Numerical errors in the eigenvalue
obtention can shuffle the ordering of the eigenlevels. In the previous figure, the
state 294 was an odd state, while here it is an even state. Also the energy
is clearly higher, and closer to previous reported results for such a level.} 
  \label{muestrasintentomejor}
\end{figure}
 

\section{Centres and Chords}

If we can trust the method described above, then we may be able to
obtain centre and chord functions from the results. The expression
for the centre function for a pure state  
reveals something that we may not be prepared
to confront, though:

\begin{equation}
  W(\xfase)=\frac{1}{(2\pi\hbar)^d}\int \rd \xi \bra{\qfase 
    +\xi/2}\hat{\rho}\ket{\qfase-\xi/2}\exp(-\ihb \pfase \cdot \xi) 
\end{equation}
In a more down to Earth  expression, for a pure state:
\begin{equation}
  W(\xfase)=\frac{1}{(2\pi\hbar)^d}\int \rd \xi \psi^*(\qfase 
  +\xi/2)\psi(\qfase-\xi/2)\exp(-\ihb \pfase \cdot \xi) 
\end{equation}

The detail in this expression is that we need \emph{twice} as better
sampling over the integrating chord $\xi$ as for the $q$-space representation
if we want to keep the same resolution. We hope that for our range of interest,
the Eigenvectors obtained have the important information, such as peaks and 
valleys adequately sampled, and nodal lines more or less clearly defined. 
Then we can \emph{oversample} the $q$ representation function with an
interpolation routine. I have chosen the ScyPy interpolation function 
with custom parameters, using a quintic spline in both axis. This enables us to 
integrate over the ``middle'' values between different points in the grid. 
The interpolated functions also produce very pretty pictures, as shown below, in 
the figure \ref{exampleinterpol}. The interpolations become slower
as the oscillations become denser, so for the first 10 Eigenfunctions it takes
some 3 minutes, while for the state 294, our old paradigmatic case, it
takes almost 30 minutes. So this interpolation time is also a measure
of complexity in the structures of the Eigenfunctions thus obtained. 

\begin{figure}[h]
  \centering
  \begin{subfigure}[b]{0.48\textwidth}
    \includegraphics[width=\textwidth]{figuras/EstadoInterpoladoMedioDenso001.png}
    \caption{state 1}
    \label{fig:gull}
  \end{subfigure}%
  % add desired spacing between images, e. g. ~, \quad, \qquad, \hfill etc.
  % (or a blank line to force the subfigure onto a new line)
  \begin{subfigure}[b]{0.48\textwidth}
    \includegraphics[width=\textwidth]{figuras/EstadoInterpoladoMedioDenso073.png}
    \caption{state 73}
    \label{fig:tiger}
  \end{subfigure}\\
  % add desired spacing between images, e. g. ~, \quad, \qquad, \hfill etc.
  % (or a blank line to force the subfigure onto a new line)
  \begin{subfigure}[b]{0.48\textwidth}
    \includegraphics[width=\textwidth]{figuras/EstadoInterpoladoMedioDenso200.png}
    \caption{state 200}
    \label{fig:mouse}
  \end{subfigure}
  \begin{subfigure}[b]{0.48\textwidth}
    \includegraphics[width=\textwidth]{figuras/EstadoInterpoladoMedioDenso294.png}
    \caption{state 294}
    \label{fig:mouse}
  \end{subfigure}
  \caption{Some Eigenfunctions as obtained by the Hernando-Zambrano method,
this time with Armadillo and Lapack on a  c++   
program, then worked out with a quintic interpolation on SciPy, before
producing the Centre and Chords Functions.} 
  \label{exampleinterpol}
\end{figure}

Obtaining the Centre Function for a fixed section in the centre space is
relative fast. If we set the centre corresponding to the first coordinate pair
to zero, that is, $(q_1,p_1)=(0,0)$, then the transform reduces to
an integration of the auto-correlations over the $\xi_{q1}$ variable and 
a Fourier Transform of the $\xi_{q2}$ variable. Numerically speaking, this is 
easy. For the same states as shown above I shall give in the next figure
the centre functions, fig. \ref{muestracentros}.


\begin{figure}[h]
  \centering
  \begin{subfigure}[b]{0.98\textwidth}
    \includegraphics[width=\textwidth]{figuras/Centros001.png}
    \caption{state 1}
    \label{fig:gull}
  \end{subfigure}%
  % add desired spacing between images, e. g. ~, \quad, \qquad, \hfill etc.
  % (or a blank line to force the subfigure onto a new line)

  \begin{subfigure}[b]{0.98\textwidth}
    \includegraphics[width=\textwidth]{figuras/Centros073.png}
    \caption{state 73}
    \label{fig:tiger}
  \end{subfigure}
  % add desired spacing between images, e. g. ~, \quad, \qquad, \hfill etc.
  % (or a blank line to force the subfigure onto a new line)

  \begin{subfigure}[b]{0.98\textwidth}
    \includegraphics[width=\textwidth]{figuras/Centros200.png}
    \caption{state 200}
    \label{fig:mouse}
  \end{subfigure}

  \begin{subfigure}[b]{0.98\textwidth}
    \includegraphics[width=\textwidth]{figuras/Centros294.png}
    \caption{state 294}
    \label{fig:mouse}
  \end{subfigure}
  \caption{The Centre Functions for the Eigenstates shown in previous figures.
Here we set $(q_1, p_1)=(0,0)$.} 
  \label{exampleinterpol}
\end{figure}

Obtaining the chord functions directly from our Eigenstates is much slower.
Of course, this was to be expected, part of the motivation of the original
work is to find more efficient methods to get this sort of Quantum Measures
without having to recurr to the extensive exact manipulation of the
State Vector. This is because there is no way to efficiently skip the double
0sampling over the points in $q$-space this time, 
even after choosing the chord that corresponds
to the first coordinate pair equal to zero. Detailing this, we have:
\begin{equation}
  \chi(\xifase)=\frac{1}{(2\pi\hbar)^d}\int \rd \qfase \psi^*(\qfase 
  +\xi/2)\psi(\qfase-\xi/2)\exp(+\ihb \qfase \cdot \xifase_p),
\end{equation}
and we set $(\xi_{q1},\xi_{p1})=(0,0)$. Then this is explicitly
\begin{equation}
  \chi(\xifase)=\frac{1}{(2\pi\hbar)^d}\int \rd q_1 \rd q_2  \psi^*(q_1, q_2+ 
  \xi_{q2}/2)\psi(q_1, q_2-\xi_{q2}/2)\exp(+\ihb q_2  \xi_{p2}).
\end{equation}
Numerically, is easier to express this as function of $2 \xifase$, so that it
coincides with our sampling density:
. Then this is explicitly
\begin{equation}
  \chi( 2 \xifase)=\frac{1}{(2\pi\hbar)^d}\int \rd q_1 \rd q_2  \psi^*(q_1, q_2+ 
  \xi_{q2})\psi(q_1, q_2-\xi_{q2})\exp(+2\ihb q_2  \xi_{p2}).
\end{equation}
The results of performing this last numerical integration are shown below, in
figure \ref{SomeChords}.

\begin{figure}[h]
  \centering
  \begin{subfigure}[b]{0.48\textwidth}
    \includegraphics[width=\textwidth]{figuras/Cuerdas001.pdf}
    \caption{state 1}
    \label{fig:gull}
  \end{subfigure}%
  % add desired spacing between images, e. g. ~, \quad, \qquad, \hfill etc.
  % (or a blank line to force the subfigure onto a new line)
  \begin{subfigure}[b]{0.48\textwidth}
    \includegraphics[width=\textwidth]{figuras/Cuerdas073.pdf}
    \caption{state 73}
    \label{fig:tiger}
  \end{subfigure}\\
  % add desired spacing between images, e. g. ~, \quad, \qquad, \hfill etc.
  % (or a blank line to force the subfigure onto a new line)
  \begin{subfigure}[b]{0.48\textwidth}
    \includegraphics[width=\textwidth]{figuras/Cuerdas200.pdf}
    \caption{state 200}
    \label{fig:mouse}
  \end{subfigure}
  \begin{subfigure}[b]{0.48\textwidth}
    \includegraphics[width=\textwidth]{figuras/Cuerdas294.pdf}
    \caption{state 294}
    \label{fig:mouse}
  \end{subfigure}
  \caption{Finally, the Chord functions obtained directly from the Interpolated
Eigenfunctions shown before. } 
  \label{SomeChords}
\end{figure}

\section{Ergodicity? What Ergodicity?}

Before I translated the code to c++, Eduardo and myself ran various
test on the higher level languages (Python and Mathematica), at lower
resolutions. We know that the results are more or less robust, and
it may happen that eigenlevels get a bit unordered by low resolution
on the $N$ and not perfectly choosen $\dt$, but still good enough to 
trust the general shape of the first 300 or so levels, out of 6400 obtained.

After visually analysing the Eigenfunctions, we had a hard felling:
for these parameters, the quantum states where absolutely not ergodic. 
That means that the higher density of probabily was acumulated 
in a very non-uniform way along the bananoid. I did a carefull
interpolation of these tests in the levels between 250 and 330, whose
energy range is $[0.7258847, 0.8325542]$. This could be very significative,
as the classical caustic in $q$-space  
has practically the same shape and size along
this parameter values, but the quantum functions are very  different in
their distribution. To make it clearer, I plotted the probability density
trazed over the $q_1=q_x$ coordinate, leaving only a probability density
function on the $q_2=:q_y$. I show some paradigmatic examples below.

\section{What can be done next?}

After visual inspection of the following figures, one can see that
the distribution (and therefore, the cumulants) of the $q_2$ vary a lot
in that energy range. We propose two courses of actions: investigage
higher levels (around 800) for which I allready have a working code.
The second one is more bold: if individual Eigenstates have too much 
structure on its configuration space representation, then we may be wrong
about the ergodic conjecture \emph{for individual eigenstates of the energy}.
We could ``wash away'' this structure by thinking on a mixed state over
a small energy range, where the distribution covers more or less evenly 
the bananoid \ldots This would be actually a strengthening of the
``weaker ergodic hypothesis'', but it may work.


\pagebreak

\begin{figure}[h]
  \centering
  \begin{subfigure}[b]{0.40\textwidth}
    \includegraphics[width=\textwidth]{figuras/EstadoInterpolado250.png}
    \caption{$\psi(\qfase)$}
    \label{fig:gull}
  \end{subfigure}\\% 
  % add desired spacing between images, e. g. ~, \quad, \qquad, \hfill etc.
  % (or a blank line to force the subfigure onto a new line)
  \begin{subfigure}[b]{0.98\textwidth}
    \includegraphics[width=\textwidth]
    {figuras/EstadoInterpolado250-CentrosWigneryProy.png}
    \caption{$W(q_y,p_y)$}
    \label{centrodenso}
  \end{subfigure}\\
  \begin{subfigure}[b]{0.40\textwidth}
    \includegraphics[width=\textwidth]
    {figuras/EstadoInterpolado250-Cuerdas-ZerosContour.pdf}
    \caption{$\chi(\xifase_2)$, Nodal Lines.}
    \label{fig:mouse}
  \end{subfigure}  
  \begin{subfigure}[b]{0.40\textwidth}
    \includegraphics[width=\textwidth]
    {figuras/CuerdasPuntos-250-ZerosContour.pdf}
    \caption{$\chi(\xifase_2)$, Nodal Lines, Pointillist Aproximation.}
    \label{fig:mouse}
  \end{subfigure}
  \caption{The 250 Eigenstate, in various representations and sections. The green line
in the subfigure \ref{centrodenso} is the partial trace over the $q_x$ values of
the expression $\psi^*(\qfase)\psi(-\qfase)$. Due to the symmetric or antisymmetric nature of the states, it is either the probability density as function of $q_y$ or the additive inverse of it. } 
  \label{Estado250}
\end{figure}

\pagebreak

\begin{figure}[h]
  \centering
  \begin{subfigure}[b]{0.40\textwidth}
    \includegraphics[width=\textwidth]{figuras/EstadoInterpolado273.png}
    \caption{$\psi(\qfase)$}
    \label{fig:gull}
  \end{subfigure}\\% 
  % add desired spacing between images, e. g. ~, \quad, \qquad, \hfill etc.
  % (or a blank line to force the subfigure onto a new line)
  \begin{subfigure}[b]{0.98\textwidth}
    \includegraphics[width=\textwidth]
    {figuras/EstadoInterpolado273-CentrosWigneryProy.png}
    \caption{$W(q_y,p_y)$}
    \label{centrodenso273}
  \end{subfigure}\\
  \begin{subfigure}[b]{0.40\textwidth}
    \includegraphics[width=\textwidth]
    {figuras/EstadoInterpolado273-Cuerdas-ZerosContour.pdf}
    \caption{$\chi(\xifase_2)$, Nodal Lines.}
    \label{fig:mouse}
  \end{subfigure}  
  \begin{subfigure}[b]{0.40\textwidth}
    \includegraphics[width=\textwidth]
    {figuras/CuerdasPuntos-273-ZerosContour.pdf}
    \caption{$\chi(\xifase_2)$, Nodal Lines, Pointillist Aproximation.}
    \label{fig:mouse}
  \end{subfigure}
  \caption{The 273 Eigenstate. } 
  \label{Estado273}
\end{figure}

\pagebreak

\begin{figure}[h]
  \centering
  \begin{subfigure}[b]{0.40\textwidth}
    \includegraphics[width=\textwidth]{figuras/EstadoInterpolado274.png}
    \caption{$\psi(\qfase)$}
    \label{fig:gull}
  \end{subfigure}\\% 
  % add desired spacing between images, e. g. ~, \quad, \qquad, \hfill etc.
  % (or a blank line to force the subfigure onto a new line)
  \begin{subfigure}[b]{0.98\textwidth}
    \includegraphics[width=\textwidth]
    {figuras/EstadoInterpolado274-CentrosWigneryProy.png}
    \caption{$W(q_y,p_y)$}
    \label{centrodenso273}
  \end{subfigure}\\
  \begin{subfigure}[b]{0.40\textwidth}
    \includegraphics[width=\textwidth]
    {figuras/EstadoInterpolado274-Cuerdas-ZerosContour.pdf}
    \caption{$\chi(\xifase_2)$, Nodal Lines.}
    \label{fig:mouse}
  \end{subfigure}  
  \begin{subfigure}[b]{0.40\textwidth}
    \includegraphics[width=\textwidth]
    {figuras/CuerdasPuntos-274-ZerosContour.pdf}
    \caption{$\chi(\xifase_2)$, Nodal Lines, Pointillist Aproximation.}
    \label{fig:mouse}
  \end{subfigure}
  \caption{The 274 Eigenstate. } 
  \label{Estado274}
\end{figure}


\pagebreak

\begin{figure}[h]
  \centering
  \begin{subfigure}[b]{0.40\textwidth}
    \includegraphics[width=\textwidth]{figuras/EstadoInterpolado280.png}
    \caption{$\psi(\qfase)$}
    \label{fig:gull}
  \end{subfigure}\\% 
  % add desired spacing between images, e. g. ~, \quad, \qquad, \hfill etc.
  % (or a blank line to force the subfigure onto a new line)
  \begin{subfigure}[b]{0.98\textwidth}
    \includegraphics[width=\textwidth]
    {figuras/EstadoInterpolado280-CentrosWigneryProy.png}
    \caption{$W(q_y,p_y)$}
    \label{centrodenso273}
  \end{subfigure}\\
  \begin{subfigure}[b]{0.40\textwidth}
    \includegraphics[width=\textwidth]
    {figuras/EstadoInterpolado280-Cuerdas-ZerosContour.pdf}
    \caption{$\chi(\xifase_2)$, Nodal Lines.}
    \label{fig:mouse}
  \end{subfigure}  
  \begin{subfigure}[b]{0.40\textwidth}
    \includegraphics[width=\textwidth]
    {figuras/CuerdasPuntos-280-ZerosContour.pdf}
    \caption{$\chi(\xifase_2)$, Nodal Lines, Pointillist Aproximation.}
    \label{fig:mouse}
  \end{subfigure}
  \caption{The 280 Eigenstate. } 
  \label{Estado280}
\end{figure}

\pagebreak

\begin{figure}[h]
  \centering
  \begin{subfigure}[b]{0.40\textwidth}
    \includegraphics[width=\textwidth]{figuras/EstadoInterpolado282.png}
    \caption{$\psi(\qfase)$}
    \label{fig:gull}
  \end{subfigure}\\% 
  % add desired spacing between images, e. g. ~, \quad, \qquad, \hfill etc.
  % (or a blank line to force the subfigure onto a new line)
  \begin{subfigure}[b]{0.98\textwidth}
    \includegraphics[width=\textwidth]
    {figuras/EstadoInterpolado282-CentrosWigneryProy.png}
    \caption{$W(q_y,p_y)$}
    \label{centrodenso273}
  \end{subfigure}\\
  \begin{subfigure}[b]{0.40\textwidth}
    \includegraphics[width=\textwidth]
    {figuras/EstadoInterpolado282-Cuerdas-ZerosContour.pdf}
    \caption{$\chi(\xifase_2)$, Nodal Lines.}
    \label{fig:mouse}
  \end{subfigure}  
  \begin{subfigure}[b]{0.40\textwidth}
    \includegraphics[width=\textwidth]
    {figuras/CuerdasPuntos-282-ZerosContour.pdf}
    \caption{$\chi(\xifase_2)$, Nodal Lines, Pointillist Aproximation.}
    \label{fig:mouse}
  \end{subfigure}
  \caption{The 282 Eigenstate. } 
  \label{Estado282}
\end{figure}


\pagebreak

\begin{figure}[h]
  \centering
  \begin{subfigure}[b]{0.40\textwidth}
    \includegraphics[width=\textwidth]{figuras/EstadoInterpolado286.png}
    \caption{$\psi(\qfase)$}
    \label{fig:gull}
  \end{subfigure}\\% 
  % add desired spacing between images, e. g. ~, \quad, \qquad, \hfill etc.
  % (or a blank line to force the subfigure onto a new line)
  \begin{subfigure}[b]{0.98\textwidth}
    \includegraphics[width=\textwidth]
    {figuras/EstadoInterpolado286-CentrosWigneryProy.png}
    \caption{$W(q_y,p_y)$}
    \label{centrodenso273}
  \end{subfigure}\\
  \begin{subfigure}[b]{0.40\textwidth}
    \includegraphics[width=\textwidth]
    {figuras/EstadoInterpolado286-Cuerdas-ZerosContour.pdf}
    \caption{$\chi(\xifase_2)$, Nodal Lines.}
    \label{fig:mouse}
  \end{subfigure}  
  \begin{subfigure}[b]{0.40\textwidth}
    \includegraphics[width=\textwidth]
    {figuras/CuerdasPuntos-286-ZerosContour.pdf}
    \caption{$\chi(\xifase_2)$, Nodal Lines, Pointillist Aproximation.}
    \label{fig:mouse}
  \end{subfigure}
  \caption{The 286 Eigenstate. } 
  \label{Estado286}
\end{figure}

\pagebreak

\begin{figure}[h]
  \centering
  \begin{subfigure}[b]{0.40\textwidth}
    \includegraphics[width=\textwidth]{figuras/EstadoInterpolado293.png}
    \caption{$\psi(\qfase)$}
    \label{fig:gull}
  \end{subfigure}\\% 
  % add desired spacing between images, e. g. ~, \quad, \qquad, \hfill etc.
  % (or a blank line to force the subfigure onto a new line)
  \begin{subfigure}[b]{0.98\textwidth}
    \includegraphics[width=\textwidth]
    {figuras/EstadoInterpolado293-CentrosWigneryProy.png}
    \caption{$W(q_y,p_y)$}
    \label{centrodenso273}
  \end{subfigure}\\
  \begin{subfigure}[b]{0.40\textwidth}
    \includegraphics[width=\textwidth]
    {figuras/EstadoInterpolado293-Cuerdas-ZerosContour.pdf}
    \caption{$\chi(\xifase_2)$, Nodal Lines.}
    \label{fig:mouse}
  \end{subfigure}  
  \begin{subfigure}[b]{0.40\textwidth}
    \includegraphics[width=\textwidth]
    {figuras/CuerdasPuntos-293-ZerosContour.pdf}
    \caption{$\chi(\xifase_2)$, Nodal Lines, Pointillist Aproximation.}
    \label{fig:mouse}
  \end{subfigure}
  \caption{The 293 Eigenstate. } 
  \label{Estado293}
\end{figure}

\pagebreak

\begin{figure}[h]
  \centering
  \begin{subfigure}[b]{0.40\textwidth}
    \includegraphics[width=\textwidth]{figuras/EstadoInterpolado294.png}
    \caption{$\psi(\qfase)$}
    \label{fig:gull}
  \end{subfigure}\\% 
  % add desired spacing between images, e. g. ~, \quad, \qquad, \hfill etc.
  % (or a blank line to force the subfigure onto a new line)
  \begin{subfigure}[b]{0.98\textwidth}
    \includegraphics[width=\textwidth]
    {figuras/EstadoInterpolado294-CentrosWigneryProy.png}
    \caption{$W(q_y,p_y)$}
    \label{centrodenso273}
  \end{subfigure}\\
  \begin{subfigure}[b]{0.40\textwidth}
    \includegraphics[width=\textwidth]
    {figuras/EstadoInterpolado294-Cuerdas-ZerosContour.pdf}
    \caption{$\chi(\xifase_2)$, Nodal Lines.}
    \label{fig:mouse}
  \end{subfigure}  
  \begin{subfigure}[b]{0.40\textwidth}
    \includegraphics[width=\textwidth]
    {figuras/CuerdasPuntos-294-ZerosContour.pdf}
    \caption{$\chi(\xifase_2)$, Nodal Lines, Pointillist Aproximation.}
    \label{fig:mouse}
  \end{subfigure}
  \caption{The 294 Eigenstate. } 
  \label{Estado294}
\end{figure}




\pagebreak

\begin{figure}[h]
  \centering
  \begin{subfigure}[b]{0.40\textwidth}
    \includegraphics[width=\textwidth]{figuras/EstadoInterpolado303.png}
    \caption{$\psi(\qfase)$}
    \label{fig:gull}
  \end{subfigure}\\% 
  % add desired spacing between images, e. g. ~, \quad, \qquad, \hfill etc.
  % (or a blank line to force the subfigure onto a new line)
  \begin{subfigure}[b]{0.98\textwidth}
    \includegraphics[width=\textwidth]
    {figuras/EstadoInterpolado303-CentrosWigneryProy.png}
    \caption{$W(q_y,p_y)$}
    \label{centrodenso273}
  \end{subfigure}\\
  \begin{subfigure}[b]{0.40\textwidth}
    \includegraphics[width=\textwidth]
    {figuras/EstadoInterpolado303-Cuerdas-ZerosContour.pdf}
    \caption{$\chi(\xifase_2)$, Nodal Lines.}
    \label{fig:mouse}
  \end{subfigure}  
  \begin{subfigure}[b]{0.40\textwidth}
    \includegraphics[width=\textwidth]
    {figuras/CuerdasPuntos-303-ZerosContour.pdf}
    \caption{$\chi(\xifase_2)$, Nodal Lines, Pointillist Aproximation.}
    \label{fig:mouse}
  \end{subfigure}
  \caption{The 303 Eigenstate. } 
  \label{Estado303}
\end{figure}





\pagebreak

\begin{figure}[h]
  \centering
  \begin{subfigure}[b]{0.40\textwidth}
    \includegraphics[width=\textwidth]{figuras/EstadoInterpolado327.png}
    \caption{$\psi(\qfase)$}
    \label{fig:gull}
  \end{subfigure}\\% 
  % add desired spacing between images, e. g. ~, \quad, \qquad, \hfill etc.
  % (or a blank line to force the subfigure onto a new line)
  \begin{subfigure}[b]{0.98\textwidth}
    \includegraphics[width=\textwidth]
    {figuras/EstadoInterpolado327-CentrosWigneryProy.png}
    \caption{$W(q_y,p_y)$}
    \label{centrodenso273}
  \end{subfigure}\\
  \begin{subfigure}[b]{0.40\textwidth}
    \includegraphics[width=\textwidth]
    {figuras/EstadoInterpolado327-Cuerdas-ZerosContour.pdf}
    \caption{$\chi(\xifase_2)$, Nodal Lines.}
    \label{fig:mouse}
  \end{subfigure}  
  \begin{subfigure}[b]{0.40\textwidth}
    \includegraphics[width=\textwidth]
    {figuras/CuerdasPuntos-327-ZerosContour.pdf}
    \caption{$\chi(\xifase_2)$, Nodal Lines, Pointillist Aproximation.}
    \label{fig:mouse}
  \end{subfigure}
  \caption{The 327 Eigenstate. } 
  \label{Estado327}
\end{figure}





\pagebreak

\begin{figure}[h]
  \centering
  \begin{subfigure}[b]{0.40\textwidth}
    \includegraphics[width=\textwidth]{figuras/EstadoInterpolado329.png}
    \caption{$\psi(\qfase)$}
    \label{fig:gull}
  \end{subfigure}\\% 
  % add desired spacing between images, e. g. ~, \quad, \qquad, \hfill etc.
  % (or a blank line to force the subfigure onto a new line)
  \begin{subfigure}[b]{0.98\textwidth}
    \includegraphics[width=\textwidth]
    {figuras/EstadoInterpolado329-CentrosWigneryProy.png}
    \caption{$W(q_y,p_y)$}
    \label{centrodenso273}
  \end{subfigure}\\
  \begin{subfigure}[b]{0.40\textwidth}
    \includegraphics[width=\textwidth]
    {figuras/EstadoInterpolado329-Cuerdas-ZerosContour.pdf}
    \caption{$\chi(\xifase_2)$, Nodal Lines.}
    \label{fig:mouse}
  \end{subfigure}  
  \begin{subfigure}[b]{0.40\textwidth}
    \includegraphics[width=\textwidth]
    {figuras/CuerdasPuntos-329-ZerosContour.pdf}
    \caption{$\chi(\xifase_2)$, Nodal Lines, Pointillist Aproximation.}
    \label{fig:mouse}
  \end{subfigure}
  \caption{The 329 Eigenstate. } 
  \label{Estado329}
\end{figure}






\bibliography{ziegos}

\end{document}