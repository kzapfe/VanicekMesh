\documentclass[a4paper,12pt]{article}

\usepackage[utf8]{inputenc}
\usepackage[spanish]{babel}
\usepackage{amsmath,amssymb}
\usepackage{graphicx}
%\usepackage{subfigure}
%\usepackage[spanish]{babel}
\usepackage{bm}
\usepackage{caption}
\usepackage{subcaption}

\usepackage[cm]{fullpage}
\usepackage[light]{antpolt}
\usepackage[T1]{fontenc}

\usepackage{float}

%Este paquete le pone una barrera a los floats
% al final de cada seccion
\usepackage[section]{placeins}


\bibliographystyle{alpha}

\newcommand{\dt}{\Delta\tau}
\newcommand{\ihb}{\frac{i}{\hbar}}
\newcommand{\xfase}{\mathbf{x}}
\newcommand{\yfase}{\mathbf{y}}
\newcommand{\qfase}{\mathbf{q}}
\newcommand{\xifase}{ {\boldsymbol{\xi}} }
\newcommand{\mufase}{ {\boldsymbol{\mu}} }
\newcommand{\Ifase}{\mathbf{I}}
\newcommand{\Pfase}{\mathbf{P}}
\newcommand{\Scat}{\mathbf{S}}
\newcommand{\Jsimp}{\mathbf{J}}
\newcommand{\Dom}{\mathbb{D}}
\newcommand{\Var}{\mathbb{M}}
\newcommand{\bra}[1]{\langle #1|}
\newcommand{\ket}[1]{|#1\rangle}
\newcommand{\braket}[2]{\langle #1|#2\rangle}
\newcommand{\Prom}[2]{\langle #1\rangle_{#2}}


\DeclareMathOperator*{\cod}{cod}
\DeclareMathOperator*{\traza}{traza}


\title{Tecnica de Hernando Segun Eduardo}
\author{Zambrano et Zapfe}


\begin{document}

Let us set things straight and formal.

La solución formal para una función de estado en MQ es la siguiente:
\begin{equation}
\Psi(q,t)=\int dq' K(q,q', t-t') \Psi(q',t'),
\end{equation}
De ahora en adelante,  $t_0=0$.
Eso es una abreviación de la siguiente expresión mas formal:
\begin{equation}\label{evolbien}
\bra{q}\ket{\Psi_t}=\int dq' \bra{q}U^t\ket{q'}\bra{q'}\ket{\Psi_0},
\end{equation}
donde hemos hecho explicito el operador identidad en la base $q'$, 
\begin{equation}
\Ifase=\int dq' \ket{q'}\bra{q'}.
\end{equation}
La naturaleza vectorial es mas evidente si pensamos en el vector de
estado, y no en la función de estado:
\begin{equation}
\ket{\Psi_t}=\int dq' U^t\ket{q'}\bra{q'}\ket{\Psi_0},
\end{equation}

Pot otra parte, formalmente el operador de evolución es:
\begin{equation}
U^t=\exp{-\ihb \hat{H} t} 
\end{equation}
Si $\hat{H}=\hat{T}+\hat{V}$ (sumando que  en general no conmutan),
aproximamos ya sea por 
\begin{equation}\label{aproxHernando}
\exp{-\ihb \hat{H} t}\approx 
 \exp{-\ihb \hat{T} t}\exp{-\ihb \hat{V}t}+O(t^2)
\end{equation}
o por algo más sofisticado
\begin{equation}\label{aproxEduardo}
\exp{-\ihb \hat{H} t}\approx
 \exp{-\ihb \hat{V}t/2}\exp{-\ihb \hat{T} t}\exp{-\ihb \hat{V}t/2}+O(t^4)
\end{equation}
Veamos el segundo caso. Insertando la aproximación en la expresión 
\ref{evolbien}, tenemos lo siguiente:
\begin{equation}\label{ketaprox1}
\bra{q}\ket{\Psi_t}  = \int dq' \bra{q} 
\exp{-\ihb \hat{V}t/2}\exp{-\ihb \hat{T} t}\exp{-\ihb \hat{V}t/} 
\ket{q'}\bra{q'}\ket{\Psi_0},
\end{equation}
dado que $exp(-\ihb \hat{V}t/2)$ es una función bien comportada de $\hat{q}$,
opera de la forma esperada sobre los eigenestados de las $q$. Entonces adquirimos
simplemente los valores esperados a la derecha e izquierda del propagador:
\begin{equation}\label{ketaprox2}
  \bra{q}\ket{\Psi_t}  = \int dq' \exp(-\ihb t/2 (V(q)+V(q'))) 
  \bra{q} \exp{-\ihb \hat{T} t}\ket{q'}\bra{q'}\ket{\Psi_0}.
\end{equation}
Lo que queda entre los brakets es el propagador de una particula
libre, asì que tenemos que ($m=1$):
\begin{equation}\label{ketaprox2}
  \bra{q}\ket{\Psi_t}  = \int dq' \exp(-\ihb t/2 (V(q)+V(q'))) 
  (\frac{ i }{2\pi\hbar t})^{d/2} \exp(\ihb 1/(2t) (q-q')^2)
  \bra{q'}\ket{\Psi_0}.
\end{equation}
Hasta aquí, nada nuevo. Viene ahora la cuestion del tiempo
imaginario. Usemos $t=-i\dt$, y supongamos que
$\Psi(q,t)$ representa un estado estacionario, es decir, 
$\Psi(q,t)=\exp{-\ihb E_n t}\psi_n(q)$.
\begin{equation}\label{ketaprox2}
  exp(-\ihb E_n \dt) \psi_n{q}  = \int dq' \exp(-\hbar \dt/2 (V(q)+V(q'))) 
  (\frac{1}{2\pi\hbar \dt})^{d/2} \exp(- \hbar 1/(2t) (q-q')^2)
  \psi_n(q').
\end{equation}

En el esquema de 
Eduardo, vamos a usar una malla uniforme de puntos sobre el
espacio de las $q$. Esto hace que nuestra identidad aproximada
tenga la forma:
\begin{equation}
\int dq' \ket{q'}\bra{q'}\approx \frac{A}{N}\sum_j \ket{q_j'}\bra{q_j'}.
\end{equation}
Donde N es el número de puntos de la malla (``grilla'') y A el área
que cubrimos en el muestreo. 

Entonces segun esto ya habríamos chingado, ya que tenemos que la ecuación
queda de la bonita forma:
\begin{equation}\label{ketaprox2}
  \exp(-\ihb E_n \dt) \psi_n{q_k}  = \frac{A}{N}\sum_j 
\exp(-\hbar \dt/2 (V(q_k)+V(q_j))) 
  (\frac{1}{2\pi\hbar \dt})^{d/2} \exp(- \hbar 1/(2t) (q_k-q_j)^2)
  \psi_n(q_j).
\end{equation}
Que ya es claramente una ecuacion de eigenvectores para el operador
\begin{equation}
\tilde{K}_{kj}=\frac{A}{N}
\exp(-\hbar \dt/2 (V(q_k)+V(q_j))) 
  (\frac{1}{2\pi\hbar \dt})^{d/2} \exp(- \hbar/(2\dt) (q_k-q_j)^2)
\end{equation}


¿Entonces cual es el problema? Veamos, para poder confiar en el 
método, tendriamos que probarlo para un sistema simple, así
que podamos checarlo con el oscilador
armónico de una dimensión. ¿Cuales son los posibles tropiezos?
La elección de $A,N$ y $\dt$. Comenzemos.

Supongamos que queremos obtener los primeros 50 eigenestados
de un OA (usemos $\hbar=1, \omega=1$). La expresión para $\tilde{K}$
quedará de la forma:
\begin{equation}
\tilde{K}_{kj}=\frac{A}{N}
(\frac{1}{2\pi\dt})^{d/2} 
\exp(-\dt/4 (q_k^2+q_j^2))
\exp(-1/(2\dt) (q_k-q_j)^2)
\end{equation}
Los valores de $q_{max}=\sqrt{2E}$ nos dan una valor de $A$:
\begin{align}
q_{max} & =\sqrt{2E_n}\\
& =\sqrt{2\hbar\omega (50+1/2)}\\
\approx{10}. 
\Rightarrow A  &> (q_{max}-(-q_{max}))=20.
\end{align}
Let us find this out. Probemos un OA de una dimensión. He aquí
los eigenniveles:

\begin{figure}[h]
\includegraphics[width=0.7\textwidth]{EigenEnergiasOA1D.pdf}
\caption{Nota como el primer valor esta un poquito mal, pero la tendencia es la
correcta. Debería de ser 0.5, y es 0.499907.}
\end{figure}


Funciono para 1D, veamos para un OA de dos dimensiones. Aquí la cosa parece estancarse.
Por mas valores que escogí para $\dt$ y probe con distintas densidades de puntos,
siempre el estado base me dio con energía negativa. Si los dos
OA tienen el mismo $\omega$, los niveles deberían de ser degenerados de orden
$n$, asi que la grafica de los niveles debe parecer algo con escalones cada
vez mas anchos, el primero de un punto, el segundo de dos, etc. Asi como queriendo
ver las cosas con mucha buena fe, usted podrá descubrir que al menos esa tendencia 
se cumple. Aquí use 10'000 puntos, es decir, 100 puntos por cada lado de la malla,
lo cual hace bastante lento el proceso de diagonalización. 


\begin{figure}[h]
\includegraphics[width=0.7\textwidth]{NoTanBien2DOA.pdf}
\caption{Dos osciladores armónicos, $N=10000, \omega_1=\omega_2=1, \hbar=1
\dt=0.0118$. El estúpido valor que obtengo para $E_0$ es $-1.779$.}
\end{figure}




\end{document}